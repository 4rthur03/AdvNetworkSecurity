
%% bare_jrnl.tex
%% V1.4
%% 2012/12/27
%% by Michael Shell
%% see http://www.michaelshell.org/
%% for current contact information.
%%
%% This is a skeleton file demonstrating the use of IEEEtran.cls
%% (requires IEEEtran.cls version 1.8 or later) with an IEEE journal paper.
%%
%% Support sites:
%% http://www.michaelshell.org/tex/ieeetran/
%% http://www.ctan.org/tex-archive/macros/latex/contrib/IEEEtran/
%% and
%% http://www.ieee.org/



% *** Authors should verify (and, if needed, correct) their LaTeX system  ***
% *** with the testflow diagnostic prior to trusting their LaTeX platform ***
% *** with production work. IEEE's font choices can trigger bugs that do  ***
% *** not appear when using other class files.                            ***
% The testflow support page is at:
% http://www.michaelshell.org/tex/testflow/


%%*************************************************************************
%% Legal Notice:
%% This code is offered as-is without any warranty either expressed or
%% implied; without even the implied warranty of MERCHANTABILITY or
%% FITNESS FOR A PARTICULAR PURPOSE! 
%% User assumes all risk.
%% In no event shall IEEE or any contributor to this code be liable for
%% any damages or losses, including, but not limited to, incidental,
%% consequential, or any other damages, resulting from the use or misuse
%% of any information contained here.
%%
%% All comments are the opinions of their respective authors and are not
%% necessarily endorsed by the IEEE.
%%
%% This work is distributed under the LaTeX Project Public License (LPPL)
%% ( http://www.latex-project.org/ ) version 1.3, and may be freely used,
%% distributed and modified. A copy of the LPPL, version 1.3, is included
%% in the base LaTeX documentation of all distributions of LaTeX released
%% 2003/12/01 or later.
%% Retain all contribution notices and credits.
%% ** Modified files should be clearly indicated as such, including  **
%% ** renaming them and changing author support contact information. **
%%
%% File list of work: IEEEtran.cls, IEEEtran_HOWTO.pdf, bare_adv.tex,
%%                    bare_conf.tex, bare_jrnl.tex, bare_jrnl_compsoc.tex,
%%                    bare_jrnl_transmag.tex
%%*************************************************************************

\documentclass{IEEEtran}

\makeatletter
\def\endthebibliography{%
	\def\@noitemerr{\@latex@warning{Empty `thebibliography' environment}}%
	\endlist
}
\makeatother

\begin{document}
%
% paper title
% can use linebreaks \\ within to get better formatting as desired
% Do not put math or special symbols in the title.
\title{Review of Cybersecurity in the Radiology Department}


\author{Arthur~Pangemanan}% <-this % stops a space



% The paper headers
\markboth{CPTE-542-A Survey Paper, October 2019}%
{Shell \MakeLowercase{\textit{et al.}}: CPTE-542-A Survey Paper, October 2019}

\maketitle

% As a general rule, do not put math, special symbols or citations
% in the abstract or keywords.
\begin{abstract}
	
	
	This paper will review the cyber threat of modern medical device within the radiology department.
\end{abstract}

% Note that keywords are not normally used for peerreview papers.
\begin{IEEEkeywords}
Cybersecurity, Security, Risk, Safety, Wireless, Medical devices.
\end{IEEEkeywords}


\section{Introduction}
\IEEEPARstart{T}{he} medical industry has change ever since the first computer were introduced. The healthcare technologies have the potential to extend, save and enhance the live of patients \cite{coventry2018cybersecurity}. Furthermore, hospitals have witnessed a proliferation of networked medical equipment in the past decade. There is an emergent trend of connection medical equipment to the hospital network for easy accessibility and manageability. Recently, securing medical devices against cyberattacks or malware outbreaks and safeguarding protected health information (PHI) stored on devices or exchanged between a device and the provider's network is a growing challenge for clinical engineers and hospital information technology (IT) professional \cite{wirth2011cybercrimes}. The number of high-profile public demonstrations of successful attacks on devices and medical networks have increased. 

Over the past few years

\section{Background}
\subsection{History Of Medical Devices}
Subsection text here.
\subsection{Implantable Medical Devices}
\subsection{Electronic Health Records}


\section{Cybersecurity in Healthcare}

\section{Types of Cyber Threats}

\section{Future Security Challenges}

The medical industry face many challenges \cite{Sametinger}.
\cite{williams2015cybersecurity}.
\cite{moses2015lack}.
\cite{ferrara2019cybersecurity}.
\cite{murphy2015cybersecurity}.
\cite{stites2016secure}.
\cite{InformationSecurityonDiagnosticImagingSystem}.
\cite{Ray}.
\cite{gerard2013cybersecurity}.
\cite{mahler2018know}.
\cite{ma2019medical}.
\cite{busdicker2017role}.
\cite{martin2017cybersecurity}.
\cite{Marwan}.
\cite{FooKune:2012:TSI:2342536.2342540}.
\cite{Almohri:2017:TMM:3204094.3204113}.
\cite{tk2013inside}.
\cite{fu2014controlling}.
\cite{tanev2015value}.
\cite{Sametinger:2015:SCM:2749359.2667218}.
\cite{tanev2015value}.
\cite{CyberSecurity}.


\section{Conclusion}
The conclusion goes here.


%--- bibliography ---
\bibliographystyle{IEEEtran}
\bibliography{reference}

% that's all folks
\end{document}


