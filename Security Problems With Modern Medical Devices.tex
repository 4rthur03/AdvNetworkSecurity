%% Support sites:
%% http://www.michaelshell.org/tex/ieeetran/
%% http://www.ctan.org/tex-archive/macros/latex/contrib/IEEEtran/
%% and
%% http://www.ieee.org/
%****************************************************************

\documentclass{IEEEtran}

\makeatletter
\def\endthebibliography{%
	\def\@noitemerr{\@latex@warning{Empty `thebibliography' environment}}%
	\endlist
}
\makeatother

\begin{document}
%
% paper title
% can use linebreaks \\ within to get better formatting as desired
% Do not put math or special symbols in the title.
\title{Review of Cybersecurity in the Radiology Department}


\author{Arthur~Pangemanan}% <-this % stops a space



% The paper headers
\markboth{CPTE-542-A Survey Paper, October 2019}%
{Shell \MakeLowercase{\textit{et al.}}: CPTE-542-A Survey Paper, October 2019}

\maketitle

% As a general rule, do not put math, special symbols or citations
% in the abstract or keywords.
\begin{abstract}
	
	
	This paper will review the cyber threat of modern medical device within the radiology department.
\end{abstract}

% Note that keywords are not normally used for peerreview papers.
\begin{IEEEkeywords}
Cybersecurity, Security, Risk, Safety, Wireless, Medical devices.
\end{IEEEkeywords}


\section{Introduction}
\IEEEPARstart{T}{he} medical industry has change ever since the first computer were introduced. The healthcare technologies have the potential to extend, save and enhance the live of patients \cite{coventry2018cybersecurity}. Furthermore, hospitals have witnessed a proliferation of networked medical equipment in the past decade. There is an emergent trend of connection medical equipment to the hospital network for easy accessibility and manageability. Recently, securing medical devices against cyberattacks or malware outbreaks and safeguarding protected health information (PHI) stored on devices or exchanged between a device and the provider's network is a growing challenge for clinical engineers and hospital information technology (IT) professional \cite{wirth2011cybercrimes}. The number of high-profile public demonstrations of successful attacks on devices and medical networks have increased. 
% New Paragraph
	Over the past few years, the question of inadequate clinical security has been gaining attention from both industry leaders and clinical practitioners.

\section{Background}
\subsection{History Of Medical Devices}
Subsection text here.
\subsection{Implantable Medical Devices}
\subsection{Electronic Health Records}


\section{Cybersecurity in Healthcare}

\section{Types of Cyber Threats}

\section{Future Security Challenges}

The medical industry face many challenges \cite{Sametinger}.
\cite{williams2015cybersecurity}.
\cite{moses2015lack}.
\cite{ferrara2019cybersecurity}.
\cite{murphy2015cybersecurity}.
\cite{stites2016secure}.
\cite{InformationSecurityonDiagnosticImagingSystem}.
\cite{Ray}.
\cite{gerard2013cybersecurity}.
\cite{mahler2018know}.
\cite{ma2019medical}.
\cite{busdicker2017role}.
\cite{martin2017cybersecurity}.
\cite{Marwan}.
\cite{FooKune:2012:TSI:2342536.2342540}.
\cite{Almohri:2017:TMM:3204094.3204113}.
\cite{tk2013inside}.
\cite{fu2014controlling}.
\cite{tanev2015value}.
\cite{Sametinger:2015:SCM:2749359.2667218}.
\cite{tanev2015value}.
\cite{CyberSecurity}.


\section{Conclusion}
The conclusion goes here.


%--- bibliography ---
\bibliographystyle{IEEEtran}
\bibliography{reference}

% that's all folks
\end{document}


