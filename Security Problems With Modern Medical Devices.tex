%% Support sites:
%% http://www.michaelshell.org/tex/ieeetran/
%% http://www.ctan.org/tex-archive/macros/latex/contrib/IEEEtran/
%% and
%% http://www.ieee.org/
%****************************************************************

\documentclass{IEEEtran}

\makeatletter
\def\endthebibliography{%
	\def\@noitemerr{\@latex@warning{Empty `thebibliography' environment}}%
	\endlist
}
\makeatother

\begin{document}
%
% paper title
% can use linebreaks \\ within to get better formatting as desired
% Do not put math or special symbols in the title.
\title{Review of Cybersecurity in the Radiology Department}


\author{Arthur~Pangemanan}% <-this % stops a space



% The paper headers
\markboth{CPTE-542-A Survey Paper, October 2019}%
{Shell \MakeLowercase{\textit{et al.}}: CPTE-542-A Survey Paper, October 2019}

\maketitle

% As a general rule, do not put math, special symbols or citations
% in the abstract or keywords.
\begin{abstract}
	
	
	This paper will review the cyber threat of modern medical device within the radiology department.
\end{abstract}

% Note that keywords are not normally used for peerreview papers.
\begin{IEEEkeywords}
Cyber-security, Security, Risk, Safety, Wireless, Medical devices.
\end{IEEEkeywords}


\section{Introduction}
\IEEEPARstart{T}{he} medical industry has change ever since the first computer were introduced. The healthcare technologies have the potential to extend, save and enhance the live of patients. Furthermore, hospitals have witnessed a proliferation of networked medical equipment in the past decade. There is an emergent trend of connection medical equipment to the hospital network for easy accessibility and manageability. As healthcare devices continue to evolve, so does the inter-connectivity. For example, it provides efficiency, error reduction, automation, and remote monitoring. Interconnected technology allows health professionals to monitor and adjust devices without the need for hospital visit or invasive procedure \cite{coventry2018cybersecurity}. With integration comes complexity and challenges in management and this protection \cite{williams2015cybersecurity}. However, interconnected technology introduces new cyber-security vulnerabilities in the same way other networked computing systems are vulnerable. Recently, securing medical devices against cyber-attacks or malware outbreaks and safeguarding protected health information (PHI) stored on devices or exchanged between a device and the provider's network is a growing challenge for clinical engineers and hospital information technology (IT) professional \cite{wirth2011cybercrimes}. The number of high-profile public demonstrations of successful attacks on devices and medical networks have increased. This fact raises the concern that inter-connectivity will directly affect clinical care and patient safety. \par
% New Paragraph
Over the past few years, the question of inadequate clinical security has been gaining attention from both industry leaders and clinical practitioners. The integration of medical devices, networking, software, and operating systems means that the relative isolation and safety of medical devices are challenged \cite{williams2015cybersecurity}. These vulnerability is also due to many manufacturers focus their efforts on innovation and functionality, with little emphasis on the network security of this devices \cite{moses2015lack}. \par
Designing a secure medical device is fundamentally different from  any other devices that only focus on safety and efficacy. Safety design decisions are based on the assumption that hazardous condition or failure occur accidentally. However, the assumption that hazardous condition or failure occurs accidentally no longer holds true as malicious attackers try to trigger hazards in devices through intentional repeated attempts \cite{Ray}. Thus manufacturer tends to not implement the necessary security check against these malicious attacks. This fact become more important as the radiology departments usually have the highest density of networked medical equipment in a hospital \cite{moses2015lack}. This paper will review the cyber threat of modern medical devices within the radiology department.

\section{Background}
With the numerous data breaches in healthcare over the last several year, it seems to be unreasonable for patients having any expectation of privacy and security in their health information. In 2012, 780,000 patients records were stolen from the State of Utah Department of Health, Department of Technology server, by an Eastern European hacker. Another at Saint Joseph's Health System in California, approximately 31,800 patients' record was made potentially available through basic Internet search engines for about a year because security settings on the system were set incorrectly \cite{murphy2015cybersecurity}. 
\subsection{Implantable Medical Devices}
A typical implantable medical devices (IMDs) monitor and treat physiological conditions within the body \cite{halperin2008security}. These devices
\subsection{Electronic Health Records}
\subsection{Radiology Devices}


\section{Cybersecurity in Radiology Department}

\subsection{Types of Cyber Threats}

\section{Future Security Challenges}

The medical industry face many challenges 


\section{Conclusion}
The conclusion goes here.


%--- bibliography ---
\bibliographystyle{IEEEtran}

\bibliography{reference}


% that's all folks
\end{document}


